\documentclass{sig-alternate}
\usepackage[latin1]{inputenc}
\usepackage{graphicx}
\usepackage{url}

\begin{document}

\conferenceinfo{GECCO'15 Student Workshop,} {July 11-15, 2015, Madrid, Spain.}
\CopyrightYear{2015}
\crdata{TBA}
\clubpenalty=10000
\widowpenalty = 10000

\title{Soft computing techniques applied to corporate and personal security}

\numberofauthors{1}
\author{
\alignauthor
P. de las Cuevas\\
\affaddr{Dept. of Computer Architecture}\\
\affaddr{and Computer Technology}\\
\affaddr{University of Granada, Spain}\\
\email{paloma@geneura.ugr.es}
}

\maketitle

\begin{abstract}
%What problem do you want to solve? Why do you want to solve it? Why
%do you think the techniques and methodology you're applying will
%solve it? Those are the three questions that you have to answer in an
%abstract - JJ
Inside a ``Bring Your Own Device'' environment, the employees can freely use their devices. This allows them mix their personal and work life, but at the same time, this environment is very insecure. The aim of this paper is defining a methodology in order to create a self-adaptative system in which the security rules/policies evolve with the environment. The system should be able to cope with huge amounts of data, applying Data Mining techniques.
Genetic Programming, and Evolutionary Algorithms should also be
applied in this system, so that the set of security rules can be
optimised and therefore cover as much insecure situations as
possible. Also, in order to better classify the unknown situations, a
novel classifier is going to be implemented.
 
The system should be also multiplatform, as there are many different
languages used for security management. %That's an implementation
                                %issue mainly. No use in the abstract
                                %- JJ 
\end{abstract}

\category{H.4}{Information Systems Applications}{Miscellaneous}
\category{G.1.6}{Mathematics of Computing}{NUMERICAL ANALYSIS}[Optimization]

\terms{}

\keywords{Data Mining, Corporate Security Policies, Evolutionary Algorithms, Machine Learning, Classification}

%
%%%%%%%%%%%%%%%%%%%%%%%%%%%%%%% INTRODUCTION %%%%%%%%%%%%%%%%%%%%%%%%%%%%%%%
%
\section{Introduction}
\label{sec:intro}

The evolution from traditional mobile phones to the so-called
smartphones has changed the way people use their devices. In addition,
security threats have evolved too \cite{gangula2013survey}, so that
new security measures have to be adopted every time a new threat
appears. More concretely, smartphones have contributed to the creation
of a \textit{Bring Your Own Device} (BYOD) scenario in which people
use their own devices at work. Despite of all the advantages that this
environment might have, it is clear that this kind of situation
creates new security challenges for the Chief Security Officer (CSO)
of a company \cite{Opp_Security11}. % because... the environment is
                                % not totally controlled, or
                                % controlled all the time, by the IT
                                % department of the company,
                                % because... - JJ

What companies normally do for establishing a list of security
measures to cope with all the security incidents which might happen,
is to build what is called a set of \textit{Corporate Security
  Policies} (CSPs). These are a set of security rules aiming at
% This happens only in BYOD environments? Also in normal
% company-issued phones or computers? How does that relate to BYOD?
% It might be obvious to you, but not to the average EC conference
% delegate - JJ
protecting company assets by defining permissions for every different
behaviour that could lead to security incidents
\cite{kaeo2003designing}. When such companies embrace a constantly
changing environment, and allow their employees to use their own
devices, the risk of having security incidents grows, even if the
employees do not have intention of attacking the company
\cite{stanton2005analysis, breivik2002abstract}. 

This paper proposes the development of a system designed to be easily
% this paper? Your thesis? Are you covering that whole topic in this
% paper? 
integrable in company servers, and capable of evolving the rules
included in a CSP by learning from past user behaviours which caused
security incidents. In order to achieve this, different techniques
% all past behaviours? Those that can be detected? - JJ
have to be applied. First, we assume that a company stores the
security incidents that have been produced, along with the context in
which they were produced. Context was defined by Abowd et
al. \cite{abowd1999towards} as ``any information that can be used to
characterize the situation of an entity''. Then, and given that this
means to analyse great quantities of data, Data Mining (DM) techniques
can help to extract useful information from it \cite{DeVel2001}, and
also from what can be considered as \textit{good behaviour} (this
means, actions that were permitted by the security rules). This
process would allow to build a classifier and to further classify new
situations. With the extracted conclusions from the performed DM
analysis, new rules can be automatically inferred. Then, as rules can
be seen as a tree whose branches are the conditions, and whose leaves
are the rule decisions, an Evolutionary Algorithm (EA) can be applied
to optimise its structure. 

The paper is structured is as follows. A brief state of the art in
company security systems and the use of DM and EAs on them is given in
the next section. Then, Section \ref{sec:methodology} explains the
overview of the architecture of the proposed system. This system will
allow the automatic creation of new security rules, as well as the
optimisation of the existing ones, for a faster response to new - and
potentially dangerous - events. Previous results obtained over a
particular type of data - URL connections - in order to evolve black
and white URL lists are presented in Section
\ref{sec:results}. Finally, conclusions and future work are shown in
Section \ref{sec:conclusions}. 

%
%%%%%%%%%%%%%%%%%%%%%%%%%%%%%%% SotA %%%%%%%%%%%%%%%%%%%%%%%%%%%%%%%
%
\section{State of the Art}
\label{sec:sota}

% 1. What do BYOD tools do?
% 2. What kind of things they don't do?
% 3. How is success or efficiency of these tools measured?
% Finally, my method will improve over those in this and that - JJ
Many companies are aware of the BYOD tendency % frase vac�a. �Cuantas
                                % compa��as lo usan? �hay alg�n dato?
                                % �Qu� mercado potencial hay? �Tu
                                % m�todo se puede usar s�lo en ese
                                % entorno? - JJ
and so they are
releasing tools for companies, as well as for devices. This way, and
more focused on the enterprise, tools by IBM \cite{IBM_tool} or Sophos
\cite{Sophos_tool} offer the CSO ways to control the devices which
enroll in the system, requiring users to employ strong passwords, for instance,
% enroll no me gusta
and also to protect the employees data by means of encryption and data
protection % at what level? - JJ
. Other tools, %for doing what? - JJ
 such as the one developed by Good's
Technology \cite{Good_tool}, adds to their features guidelines for the
CSOs to develop good CSPs. However, not one of the reviewed tools has
the ability of inferring new rules or refining the existing ones. 

On the device side, the most powerful solution % for doing what? BYOD?
                                % - JJ
seems to be to directly
use a phone which has been developed with data security in mind such as the BlackPhone
\cite{Blackphone_site}. It has its own Android-based operating system,
called \textit{PrivatOS}, which includes a privacy-focused application
store (called \textit{Silent Store}) that takes care of the problem of
applications which ask for certain permissions that can lead to data
loss \cite{gangula2013survey}. This BlackPhone also allows a remote
% The problem you're trying to prevent is data loss? How big is that
% problem? - JJ
wiping of the data if the device is lost or stolen. The main
disadvantage of this solution is either the enterprise having to make
an investment and buy these smartphones to the employees (which, in
fact, is against the BYOD philosophy), or to make employees buy them,
so they cannot use the device they already have.  

Finally, and though to be an extension of Android devices, two main tools can be found in the market. One tool developed by Samsung \cite{Samsung_tool}, for Samsung devices, and another one developed by Google itself, called Android Work \cite{AndroidWork_site}. Both of them have most of the same advantages as the blackphone, with the addition of an extension for CSOs. This means that Samsung, as well as Google, provide security tools both at device and server side. More precisely, Android work follows the way of working that Blackberry phones started with Blackberry balance \cite{Blackberry_tool}, which stands up for having \textit{work} applications and \textit{personal} applpications. This is called a ``dual-persona'' smartphone \cite{AndroidWork_review}. However, with regard to CSPs, neither of these tools specify ``self-adaptation' ' as a feature. They offer policy management, but still the do not analyse the system information for security rules evolution purposes.

With respect to the application of Data Mining to extract information from big amounts of data, this has been done since the nineties \cite{agrawal1995mining, ester1996density}. More specifically, DM has been widely used for security purposes, as it can be applied in computer forensics. O. de Vel studied the application of DM techniques to identify authors of malicious e-mails in \cite{DeVel2001}, and for performing ``offender profiling'' in relation to computer security attacks in \cite{abraham2002investigative}. Yet, the system this paper proposed is focused in doing this kind of analysis but then to look for similarities with the new incoming events, so that a decision can be made in case they are dangerous. Classification methods are also applied in the security field. For instance, Blanzieri and Bryl \cite{blanzieri2008} present a review on a variety of spam filtering methods, and compare them, reaching the conclusion of that they are successful in general, but yet insufficient. This is why implementing a self-adaptive system such as the one this paper proposes can be good for other security applications and not only spam classification.

As for the works related with the users' information and behaviour, and the management (and adaptation) of the set of Corporate Security Policies, many can be found in literature. For instance, P.G. Kelley et al. \cite{user-controllable_learning_08} presented a method named \textit{user-controllable policy learning} in which the user gives feedback to the system every time it applies a security policy. Then, these policies can be refined according to that feedback to be more accurate with respect to what the users need. This approach could be useful for adding information to the system, and therefore perform a deeper analysis to extract more accurate conclusions, and finally create better rules.
Then, taking into account how much information can be gathered from social networks, Danezis in \cite{inferring_policies_socialnetworks_09} defined a system able to infer privacy-related restrictions, enhancing user's privacy, by applying Machine Learning techniques on a social network environment. Again, this is another interesting approach. However, this paper focuses on CSPs, related to companies, more than on personal life of individuals.

In the same line, Lim et al. proposed a system \cite{lim2008mls, lim2008policy} which evolves a set of computer security policies by means of Genetic Progamming, gathering knowledge from the user's feedback like in \cite{user-controllable_learning_08}. Furthermore, Suarez-Tangil et al. \cite{suarez2009automatic} take the same approach as Lim et al., but also bringing event correlation in. These two latter author's works are interesting for this paper, though they are not focused on company CSPs.

Next section describes the methodology which the system proposed in this work will follow. The development of this system is supported by previous experiment that are explained in Section \ref{sec:results}.

%
%%%%%%%%%%%%%%%%%%%%%%%%%%%%%%% METHODOLOGY %%%%%%%%%%%%%%%%%%%%%%%%%%%%%%%
%
\section{Methodology}
\label{sec:methodology}

The proposed system is intended to be placed inside the server of the company which wants to add a rule-refinement feature to its security system. Furthermore, this self-adaptive system can be seen as a feature extension of the tools described in the previous section. Figure \ref{fig:krs} shows an overview of the architecture components of the proposed system.

In order to understand the flow of information, it must be noted that the \textit{database} represents only the part inside the company server where the needed data is stored. This also contributes to preserve privacy, for the system would only have rights to access some piece of information. The following subsections describe the two main components of this system: the \textit{data mining analyser}, and the \textit{rule treatment} component, which will use Evolutionary Algorithms for creating and evolving security rules.

\begin{figure*}
  \begin{center}
    \includegraphics[width=0.5\textwidth]{./img/KRSgecco.png}
    \caption{Architecture overview of the proposed system, whith its inner components.}
    \label{fig:krs}
  \end{center}
\end{figure*}

\subsection{Data mining analyser}
\label{subsec:datamining}

This component wil be in charge of taking the desired `raw' data from the database, and processing it to remove errors or non-valid values, in order to obtain a dataset to be used in the rule treatment process. For instance, duplicated data or unknown values are considered as data that should be removed. Considered data corresponds to events (and their related information/context) produced by users' interactions with the system.
Then, the preprocessing component will be devoted to `prepare' this data for the application of further techniques such as pattern mining \cite{han2007frequent}. Patter mining allows the identification of non-frequent or anomalous patterns, since these are suspicious, and thus, could be of interest to be checked by the Chief Security Officer.

The next subcomponent performs tasks like feature selection \cite{guyon2003introduction}, which consists of choosing the most important data features/variables in order to reduce the dataset weight. Also, new features can be created by extracting meta-information from the existing ones. These two steps are mainly done for improving the performance of the classification stage. Then the subcomponent uses classification algorithms \cite{witten2005data}, i.e., it trains models (classifiers) able to associate every pattern in the dataset to a class. This way, the built classifier can assign a class to further incoming patterns.  In this case, the class will be the ``decision'' taken, which means that if the incoming user action is too similar to past dangerous patterns, it will be rejected ot denied.

\subsection{Rule treatment}
\label{subsec:ruletreatment}

This component will be focused in creating new rules and will also work with the existing CSPs.
It will globally perform three different tasks over them. First, it will take the set of classification rules from the previous component, and will merge or compare them with the existing ones, suggesting a first set of new (unrefined) rules. Then, it will analyse the existing rules in order to remove, from the created set in the first step, those which might be redundant. This is done for maintaining correctness and coherence in the system. Finally, taking advantage of the decision tree structure nature of the rules, the system will consider using Genetic Programming \cite{koza1992genetic}, as the kind of EA used for optimising tree-based structures. The final set of rules will be presented to the CSO of the company, then accept or reject them. This acceptance or rejection of rule process is itself a `feedback' from wich the system can learn.

%
%%%%%%%%%%%%%%%%%%%%%%%%%%%%%%% PRELIMINARY RESULTS %%%%%%%%%%%%%%%%%%%%%%%%%%%%%%%
%
\section{Preliminary Results}
\label{sec:results}

As mentioned, the decision of implementing this system is preceded by the result obtained in \cite{mora14:urls}. 

%
%%%%%%%%%%%%%%%%%%%%%%%%%%%%%%% CONCLUSIONS %%%%%%%%%%%%%%%%%%%%%%%%%%%%%%%
%
\section{Conclusions}
\label{sec:conclusions}

%
%%%%%%%%%%%%%%%%%%%%%%%%%%%%%%% ACKNOWLEDGEMENTS %%%%%%%%%%%%%%%%%%%%%%%%%%%%%%%
%
\section{Acknowledgements}

\bibliographystyle{plain}
\bibliography{sc_sec}

\end{document}
