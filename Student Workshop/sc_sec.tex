\documentclass{sig-alternate}
\usepackage[latin1]{inputenc}
\usepackage{graphicx}
\usepackage{url}

\begin{document}

\conferenceinfo{GECCO'15 Student Workshop,} {July 11-15, 2015, Madrid, Spain.}
\CopyrightYear{2015}
\crdata{TBA}
\clubpenalty=10000
\widowpenalty = 10000

\title{Soft computing techniques applied to corporate and personal security}

\numberofauthors{1}
\author{
\alignauthor
P. de las Cuevas\\
\affaddr{Dept. of Computer Architecture}\\
\affaddr{and Computer Technology}\\
\affaddr{University of Granada, Spain}\\
\email{paloma@geneura.ugr.es}
}

\maketitle

\begin{abstract}
Inside a "Bring Your Own Device" environment, the employees can freely use their devices. This allows them mix their personal and work life, but at the same time, this environment is very unsecure. The aim is to create a self-adaptative system in which the security rules/policies evolve with the environment. The system will be able to cope with huge amounts of data, applying Data Mining techniques.
Genetic Programming, and Evolutionary Algorithms are going to be applied in this system, so that the set of security rules can be optimised and cover as much unsecure situations as possible. Also, in order to better classify the unknown situations, a novel classifier is going to be implemented.
The system will be also multiplatform, as there are many different languages used for security management.
\end{abstract}

\category{H.4}{Information Systems Applications}{Miscellaneous}
\category{G.1.6}{Mathematics of Computing}{NUMERICAL ANALYSIS}[Optimization]

\terms{}

\keywords{}

%
%%%%%%%%%%%%%%%%%%%%%%%%%%%%%%% INTRODUCTION %%%%%%%%%%%%%%%%%%%%%%%%%%%%%%%
%
\section{Introduction}
\label{sec:intro}

%
%%%%%%%%%%%%%%%%%%%%%%%%%%%%%%% SotA %%%%%%%%%%%%%%%%%%%%%%%%%%%%%%%
%
\section{State of the Art}
\label{sec:sota}

%
%%%%%%%%%%%%%%%%%%%%%%%%%%%%%%% METHODOLOGY %%%%%%%%%%%%%%%%%%%%%%%%%%%%%%%
%
\section{Methodology}
\label{sec:methodology}

%
%%%%%%%%%%%%%%%%%%%%%%%%%%%%%%% PRELIMINARY RESULTS %%%%%%%%%%%%%%%%%%%%%%%%%%%%%%%
%
\section{Preliminary Results}
\label{sec:results}

%
%%%%%%%%%%%%%%%%%%%%%%%%%%%%%%% CONCLUSIONS %%%%%%%%%%%%%%%%%%%%%%%%%%%%%%%
%
\section{Conclusions}
\label{sec:conclusions}

%
%%%%%%%%%%%%%%%%%%%%%%%%%%%%%%% ACKNOWLEDGEMENTS %%%%%%%%%%%%%%%%%%%%%%%%%%%%%%%
%
\section{Acknowledgements}

\bibliographystyle{plain}
\bibliography{sc_sec}
\end{document}
