\documentclass{sig-alternate}
\usepackage[latin1]{inputenc}
\usepackage{graphicx}
\usepackage{url}

\begin{document}

\conferenceinfo{GECCO'15 Student Workshop,} {July 11-15, 2015, Madrid, Spain.}
\CopyrightYear{2015}
\crdata{TBA}
\clubpenalty=10000
\widowpenalty = 10000

\title{Soft computing techniques applied to corporate and personal security}

\numberofauthors{1}
\author{
\alignauthor
P. de las Cuevas\\
\affaddr{Dept. of Computer Architecture}\\
\affaddr{and Computer Technology}\\
\affaddr{University of Granada, Spain}\\
\email{paloma@geneura.ugr.es}
}

\maketitle

\begin{abstract}
Inside a ``Bring Your Own Device'' environment, the employees can freely use their devices. This allows them mix their personal and work life, but at the same time, this environment is very insecure. The aim of this paper is defining a methodology in order to create a self-adaptative system in which the security rules/policies evolve with the environment. The system should be able to cope with huge amounts of data, applying Data Mining techniques.
Genetic Programming, and Evolutionary Algorithms should also be applied in this system, so that the set of security rules can be optimised and therefore cover as much insecure situations as possible. Also, in order to better classify the unknown situations, a novel classifier is going to be implemented.
The system should be also multiplatform, as there are many different languages used for security management.
\end{abstract}

\category{H.4}{Information Systems Applications}{Miscellaneous}
\category{G.1.6}{Mathematics of Computing}{NUMERICAL ANALYSIS}[Optimization]

\terms{}

\keywords{Data Mining, Corporate Security Policies, Evolutionary Algorithms, Machine Learning, Classification}

%
%%%%%%%%%%%%%%%%%%%%%%%%%%%%%%% INTRODUCTION %%%%%%%%%%%%%%%%%%%%%%%%%%%%%%%
%
\section{Introduction}
\label{sec:intro}

The evolution from traditional mobile phones to the so-called \textit{smartphones} has changed the way people use their devices. In addition, security threats have evolved too \cite{gangula2013survey}, so that new security measures have to be adopted every time a new threat appears. More concretely, \textit{smartphones} have contributed to the creation of a \textit{Bring Your Own Device} (BYOD) scenario in which people use their own devices at work. Despite of all the advantages that this environment might have, it is clear that this kind of situation creates new security challenges for the Chief Security Officer (CSO) of a company \cite{Opp_Security11}.

What companies normally do for establishing a list of security measures to cope with all the security incidents which might happen, is to build what is called a set of \textit{Corporate Security Policies} (CSPs). These are a set of security rules aiming at protecting company assets by defining permissions for every different behaviour that could lead to security incidents \cite{kaeo2003designing}. When such companies embrace a constantly changing environment, and allow their employees to use their own devices, the risk of having security incidents grow, even if the employees do not have intention of attacking the company \cite{stanton2005analysis, breivik2002abstract}.

This paper proposes the development of a system, easy to integrate in company servers, and capable of evolving the rules included in a CSP by learning from past user behaviours which caused security incidents. In order to achieve this, different techniques have to be applied. First, we assume that a company stores the security incidents that have been produced, along with the context\footnote{Defined by Abowd et al. \cite{abowd1999towards} as ``any information that can be used to characterize the situation of an entity''.} in which they were produced. Then, and given that this means to analyse great quantities of data, Data Mining (DM) techniques can help to extract useful information from it \cite{DeVel2001}, and also from what can be considered as \textit{good behaviour} (this means, actions that were permitted by the security rules). This process would allow to build a classifier and to further classify new situations. With the extracted conclusions from the performed DM analysis, new rules can be automatically inferred. Then, as rules can be seen as a tree whose branches are the conditions, and whose leaves are the rule decisions, an Evolutionary Algorithm (EA) can be applied to optimise its structure.

The paper is structured is as follows. A brief state of the art in company security systems and the use of DM and EAs on them is given in the next section. Then, Section \ref{sec:methodology} explains the overview of the architecture of the proposed system. Previous results obtained over a particular type of data - URL connections - in order to evolve black and white URL lists are presented in Section \ref{sec:results}. Finally, conclusions are shown in Section \ref{sec:conclusions}.

%
%%%%%%%%%%%%%%%%%%%%%%%%%%%%%%% SotA %%%%%%%%%%%%%%%%%%%%%%%%%%%%%%%
%
\section{State of the Art}
\label{sec:sota}

Many companies are aware of the BYOD tendency and so they are releasing tools for companies, as well as for devices. This way, and more focused on the enterprise, tools by IBM \cite{IBM_tool} or Sophos \cite{Sophos_tool} offer the CSO ways to control the devices which enroll in the system, requiring them strong passwords, for instance, and also to protect the employees data by means of encryption and data protection. Other tools, such as the one developed by Good's Technology \cite{Good_tool}, adds to their features guidelines for the CSOs to develop good CSPs. However, not one of the reviewed tools has the ability of inferring new rules or refining the existing ones.

On the device side, the most powerful solution seems to be to directly use a phone which has been developed for data security, the blackphone \cite{Blackphone_site}. It has its own Android-based Operating System, called \textit{PrivatOS}, which includes a privacy-focused application store (called \textit{Silent Store}) that takes care of the problem of applications which ask for certain permissions that can lead to data loss \cite{gangula2013survey}. This blackphone also allows a remote wiping of the data if the device is lost or stolen. The main disadvantage of this solution is either the enterprise having to make an investment and buy these \textit{smartphones} to the employees (which, in fact, is against the BYOD philosophy), or to make employees buy them, so they cannot use the device they already have.

Finally, and though to be an extension of Android devices, we can find one tool developed by Samsung \cite{Samsung_tool}, for Samsung devices, and another developed by Google itself, called Android Work \cite{AndroidWork_site}.

%
%%%%%%%%%%%%%%%%%%%%%%%%%%%%%%% METHODOLOGY %%%%%%%%%%%%%%%%%%%%%%%%%%%%%%%
%
\section{Methodology}
\label{sec:methodology}

%
%%%%%%%%%%%%%%%%%%%%%%%%%%%%%%% PRELIMINARY RESULTS %%%%%%%%%%%%%%%%%%%%%%%%%%%%%%%
%
\section{Preliminary Results}
\label{sec:results}

%
%%%%%%%%%%%%%%%%%%%%%%%%%%%%%%% CONCLUSIONS %%%%%%%%%%%%%%%%%%%%%%%%%%%%%%%
%
\section{Conclusions}
\label{sec:conclusions}

%
%%%%%%%%%%%%%%%%%%%%%%%%%%%%%%% ACKNOWLEDGEMENTS %%%%%%%%%%%%%%%%%%%%%%%%%%%%%%%
%
\section{Acknowledgements}

\bibliographystyle{plain}
\bibliography{sc_sec}
\end{document}
